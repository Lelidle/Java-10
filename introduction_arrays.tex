\documentclass[fontsize=12pt]{article}
\usepackage[utf8]{inputenc}
\usepackage[T1]{fontenc}
\usepackage[german]{babel}
\usepackage{amsmath}
\usepackage{amsthm}
\usepackage{amsfonts}
\usepackage{amssymb}
\usepackage{minted}
\usepackage{tikz}
\usepackage{pgfplots}
\usepackage[top=2cm, bottom=2cm, left=2cm, right=2cm, headheight=1.5cm]{geometry}
\usepackage{fancyhdr}
\usepackage{mdframed}
\usemintedstyle{emacs}

\definecolor{purp}{HTML}{9A72AC}
\definecolor{re}{HTML}{FC6255}
\definecolor{gre}{HTML}{83C167}
\definecolor{blu}{HTML}{58C4DD}
\definecolor{shadecolor}{rgb}{0.85,0.85,0.85}
\definecolor{bg}{rgb}{0.95,0.95,0.95}
\setlength{\parindent}{0em} 

\BeforeBeginEnvironment{minted}{\begin{mdframed}[linewidth =2 ,backgroundcolor=bg , linecolor=black, linewidth=0.5]}
\AfterEndEnvironment{minted}{\end{mdframed}}

\newenvironment{defi}[1]{
    \begin{shaded*}
    \textbf{Definition #1} \\
}{
    \end{shaded*}
}

\newcommand{\bsp}{\textbf{Beispiel}:}
%\newcommand{\task}{\textbf{Aufgabe}:}

\newcommand{\bol}[1]{\textbf{#1}}
\newcommand{\q}[1]{\glqq #1\grqq}
\newcommand{\DODO}[1]{\textbf{\textcolor{red}{DODO:}} #1 \\ \begin{center}\includegraphics[scale=0.2]{../../media/dodo.jpg} \end{center}}

\newenvironment{task}[1]{
    \begin{shaded*}
    \textbf{Aufgabe #1}:
}{
    \end{shaded*}
}




\fancyhead[L]{\LARGE\textbf{Felder (arrays)}}
\fancyhead[R]{\Large \textbf{Datum:} \hspace{2cm}}

\begin{document}
\setlength{\headsep}{12pt}
Es kommt häufig vor, dass in einem bestimmten Kontext nicht nur eine Variable, sondern dutzende oder hunderte gebraucht werden, um manche Dinge zu modellieren, Beispiele:
\begin{itemize}
    \item Tic-Tac-Toe-Feld, Schachbrett, allgemein: Spielfeld.
    \item Liste von Zufallszahlen. 
    \item Noten einer Klasse bei einer Arbeit, etc. 
\end{itemize}

\begin{defi}{Array}
Ein Feld (array) ist eine Datenstruktur, in der mehrere Werte (gleichen Typs) gespeichert werden können.
\end{defi}

\textbf{Hinweise:}
\begin{itemize}
    \item Der englische Begriff \q{array} hat sich im Deutschen bereits so eingebürgert, dass außerhalb der Schule so gut wie niemand von \q{Feldern} spricht. 
    \item Streng genommen ist die Voraussetzung \q{gleicher Typ} in Java nicht für die Speicherung notwenidg, wir beschränken uns aber zunächst darauf. 
\end{itemize}



\end{document}